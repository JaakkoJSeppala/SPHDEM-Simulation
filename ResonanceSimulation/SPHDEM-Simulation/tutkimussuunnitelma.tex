\documentclass[utf8, finnish]{gradu3}
\usepackage{lipsum} % Poista tämä rivi lopullisesta versiosta (käytetty esimerkkitekstiin)

% Metatiedot
\title{Research Plan: SPH-DEM Modelling of Granular Dampers for Ballast Tank Sloshing Mitigation}
\translatedtitle{Tutkimussuunnitelma: SPH-DEM-mallinnus granular dampereista ballastitankkien sloshing-ilmiön vaimentamiseksi}
\author{Jaakko Seppälä}
\supervisor{Sampsa Kiiskinen}
\supervisor{Tytti Saksa}
\studyline{Tietotekniikan tutkinto-ohjelma}
\tiivistelma{
Tämä tutkimussuunnitelma esittelee SPH-DEM-mallinnuksen käytön granular dampereiden tehokkuuden arvioimiseksi ballastitankkien sloshing-ilmiön vaimentamisessa. Työ keskittyy numeeristen simulaatioiden avulla optimaalisten parametrien etsintään ja energian dissipaation mallintamiseen.
}
\avainsanat{SPH, DEM, granular damper, sloshing, ballast tank, numeerinen mallinnus}
\keywords{SPH, DEM, granular damper, sloshing, ballast tank, numerical modelling}
\contactinformation{jaakko.j.seppala@student.jyu.fi}

\begin{document}


\maketitle

% Sisällysluettelo
% \tableofcontents

% 1. Johdanto
\section{Introduction}
\label{sec:introduction}
Sloshing in ballast tanks is a critical phenomenon that can lead to structural fatigue and reduced operational lifetime of ships. Granular dampers offer a passive solution to mitigate sloshing-induced loads, but their effectiveness depends on complex interactions between fluid and granular materials. This research aims to develop a coupled \textbf{Smoothed Particle Hydrodynamics (SPH)} and \textbf{Discrete Element Method (DEM)} model to simulate and optimize granular dampers for sloshing mitigation.

\subsection{Scientific and Practical Significance}
The scientific contribution of this work lies in the novel combination of SPH and DEM to model the multiphase system of fluid and granular materials. Practically, the results can provide guidelines for designing granular dampers in ballast tanks, improving the safety and durability of marine structures.

\subsection{Background}
Sloshing is a resonant fluid motion that occurs in partially filled tanks, causing high-pressure impacts on tank walls. Granular dampers dissipate energy through particle collisions and friction, but their design requires accurate numerical tools. SPH is ideal for capturing free-surface flows, while DEM excels in modeling granular dynamics. Combining these methods enables a realistic representation of the coupled system.

% 2. Kirjallisuuskartoitus
\section{Literature Review}
\label{sec:literature_review}

\subsection{Search Methodology}
The literature search was conducted using \textbf{Scopus} and \textbf{Web of Science} databases with keywords such as "SPH-DEM coupling," "granular damper," "sloshing mitigation," and "ballast tank." Key sources include:
\begin{itemize}
    \item Monaghan (2005) for SPH fundamentals.
    \item Pöschel \& Schwager (2005) for DEM and granular dynamics.
    \item Heckel et al. (2012) and Avdić et al. (2024) for granular damper applications.
\end{itemize}

\subsection{Key Findings}
Previous studies have demonstrated the effectiveness of granular dampers in reducing vibrations \parencite{heckel2012}. However, most research focuses on either SPH or DEM separately, with limited work on their coupled application in sloshing scenarios. This research addresses this gap by integrating SPH and DEM to model the fluid-granular interaction in ballast tanks.

\subsection{Research Gap}
While SPH and DEM have been used individually for fluid and granular simulations, their combined use for sloshing mitigation in ballast tanks remains underexplored. This study aims to fill this gap by:
\begin{itemize}
    \item Developing a coupled SPH-DEM model for sloshing and granular damper interaction.
    \item Investigating the influence of granular damper parameters (e.g., particle size, filling ratio) on energy dissipation.
\end{itemize}

% 3. Tutkimusaihe ja tutkimuskysymykset
\section{Research Topic and Questions}
\label{sec:research_questions}
The primary research topic is:
\textbf{"SPH-DEM Modelling of Granular Dampers for Ballast Tank Sloshing Mitigation."}

The specific research questions are:
\begin{enumerate}
    \item How can the SPH-DEM model accurately capture the interaction between sloshing fluid and granular dampers?
    \item What are the optimal parameters (e.g., particle size, damper geometry) for maximizing energy dissipation?
    \item Can the model be used to optimize granular damper design for real-world applications?
\end{enumerate}

% 4. Tutkimusstrategia ja menetelmä
\section{Research Strategy and Methodology}
\label{sec:methodology}
The research employs a \textbf{numerical modelling} approach using the coupled SPH-DEM method. This methodology was chosen because:
\begin{itemize}
    \item SPH is highly effective for simulating free-surface flows like sloshing \parencite{monaghan2005}.
    \item DEM accurately models the discrete nature of granular materials \parencite{poschel2005}.
    \item The coupling of SPH and DEM allows for a comprehensive analysis of the fluid-granular system.
\end{itemize}

\subsection{Software and Tools}
The simulations will be performed using:
\begin{itemize}
    \item \textbf{DualSPHysics} for SPH simulations.
    \item \textbf{LIGGGHTS} for DEM simulations.
    \item \textbf{Python} for post-processing and data analysis.
\end{itemize}

% 5. Aineiston keruun suunnittelu
\section{Data Collection Planning}
\label{sec:data_collection}
The data will consist of simulation results, including:
\begin{itemize}
    \item Pressure distributions on tank walls.
    \item Energy dissipation rates.
    \item Particle velocity and collision data.
\end{itemize}

\subsection{Ethical Considerations}
This research involves only numerical simulations, so no personal data or ethical approvals are required. The work adheres to the \textbf{University of Jyväskylä's guidelines for good scientific practice}.

% 6. Aineiston keruu
\section{Data Collection Process}
\label{sec:data_collection_process}
The data collection will proceed as follows:
\begin{enumerate}
    \item \textbf{Simulations}: SPH-DEM simulations will be run on the University of Jyväskylä's computing cluster during Fall 2025.
    \item \textbf{Parameters}: The simulations will vary granular damper parameters (e.g., particle size, filling ratio) to identify optimal configurations.
    \item \textbf{Storage}: Results will be stored in CSV format and backed up on the university's secure servers.
\end{enumerate}

% 7. Aineiston analyysi
\section{Data Analysis}
\label{sec:data_analysis}
The analysis will focus on:
\begin{itemize}
    \item Comparing pressure distributions with and without granular dampers.
    \item Calculating energy dissipation rates for different damper configurations.
    \item Identifying correlations between damper parameters and sloshing mitigation effectiveness.
\end{itemize}
\textbf{Tools}: Python (with libraries such as NumPy, Pandas, and Matplotlib) will be used for data analysis and visualization.

% 8. Odotetut tulokset
\section{Expected Results}
\label{sec:expected_results}
The expected outcomes include:
\begin{itemize}
    \item A validated SPH-DEM model for sloshing and granular damper interaction.
    \item Insights into the optimal design parameters for granular dampers.
    \item Guidelines for applying the model in industrial settings.
\end{itemize}

% 9. Johtopäätökset
\section{Conclusions}
\label{sec:conclusions}
This research will contribute to the understanding of granular dampers' role in sloshing mitigation. The results are expected to:
\begin{itemize}
    \item Provide a numerical tool for designing granular dampers.
    \item Highlight the most influential parameters for energy dissipation.
    \item Offer recommendations for future experimental validation.
\end{itemize}

% 10. Lähdeluettelo
\printbibliography

% 11. Liitteet
% \appendix
% \section{Appendices}
% \label{sec:appendices}
% \subsection{Simulation Settings}
% Example simulation parameters and configurations will be included here.

% \subsection{Sample Code}
% Relevant code snippets for SPH-DEM coupling will be provided.

\end{document}
