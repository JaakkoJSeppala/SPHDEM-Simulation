\documentclass[12pt,a4paper]{report}

% Packages
\usepackage{amsmath, amssymb}
\usepackage{graphicx}
\usepackage{hyperref}
\usepackage{geometry}
\geometry{margin=1in}



%-------------------------
% Title Page
%-------------------------
\title{SPH--DEM Modelling of Granular Dampers for Ballast Tank Sloshing Mitigation}
\author{Jaakko Seppälä}
\date{Master's Thesis \\ University of Jyväskylä \\ 2025}
\begin{document}
\maketitle

%-------------------------
% Abstract
%-------------------------
\begin{abstract}
This thesis investigates the use of Smoothed Particle Hydrodynamics (SPH) coupled with the Discrete Element Method (DEM) to model granular dampers for mitigating sloshing in ballast tanks. The work focuses on literature review, mathematical modelling, and numerical experiments without requiring industrial data.
\end{abstract}

%-------------------------
\tableofcontents

%-------------------------
\chapter{Introduction}
%-------------------------
Fluid sloshing in partially filled tanks is a critical issue in naval and offshore engineering. When a ship or floating structure is subjected to wave-induced motions, the free-surface oscillations of liquid cargo or ballast water can resonate with the vessel’s natural frequencies. This may amplify dynamic loads, reduce stability, and compromise both structural integrity and operational safety (White, \emph{Fluid Mechanics}, 8th ed.). Sloshing phenomena are especially important in ballast tanks, which are essential for ship stability but also prone to resonant free-surface oscillations.

Traditional sloshing mitigation methods include the installation of internal baffles, tuned liquid dampers (TLDs), and passive anti-slosh devices. While effective, these solutions often involve structural modifications, increased weight, or reduced cargo capacity. An alternative approach is the use of granular dampers, in which dissipative particle–particle and particle–wall collisions convert kinetic energy into heat. Such devices have been successfully studied in the context of vibration attenuation (Pöschel and Schwager, \emph{Computational Granular Dynamics}), yet their potential for mitigating sloshing in marine applications remains largely unexplored.

The research gap lies in the limited application of granular dampers to fluid–structure interaction problems involving large-scale free-surface flows. Existing studies focus either on sloshing suppression by traditional fluid-based devices or on the performance of granular dampers under mechanical vibrations. Few investigations attempt to couple both systems in a unified framework suitable for ballast tank environments.

The aim of this thesis is to develop and explore a coupled SPH--DEM framework for simulating granular dampers inside ballast tanks subjected to sloshing. The scope includes (i) a literature review of sloshing mitigation and granular damping methods, (ii) derivation of the mathematical models for SPH and DEM, (iii) implementation of the coupling scheme, and (iv) numerical case studies to assess the damping efficiency of granular media in different configurations.

The research questions guiding this thesis are:
\begin{enumerate}
    \item How accurately can an SPH--DEM model reproduce fluid–particle interactions relevant to ballast tank sloshing?
    \item What role do granular properties (particle size, density, friction) and fill fractions play in sloshing mitigation?
    \item How effective are granular dampers compared to conventional baffles or tuned liquid dampers in reducing sloshing-induced loads?
    \item What are the computational and methodological limitations of SPH--DEM when applied to large-scale marine problems?
\end{enumerate}

%-------------------------
\chapter{Literature Review}
%-------------------------


\section{Granular Dampers}

Granular (or particle) dampers are passive devices consisting of loose granular material contained inside an enclosure attached to a vibrating structure. Energy is dissipated primarily through inelastic particle--particle and particle--wall collisions and frictional contacts; the collective motion of the particles converts coherent mechanical energy into heat and small-scale granular agitation. This behaviour and the underlying contact mechanics are described in detail, for example, in Pöschel and Schwager's \emph{Computational Granular Dynamics} (2005).

Experimental and numerical studies show that granular dampers operate in distinct dynamic regimes depending on excitation amplitude and frequency. At small excitations, particles move collectively (``collective collision'' regime), while at larger amplitudes the particle bed can fluidise and individual collisions dominate. This leads to very different energy dissipation characteristics. Sack, Heckel and Pöschel (2020) reported a sharp transition between these regimes in microgravity experiments.

Key design parameters that govern damping performance include particle size and size distribution, particle density, particle shape (sphericity), friction and restitution coefficients, fill fraction (the ratio of particle volume to cavity volume), and enclosure geometry. Numerical DEM studies, such as Chen and Wang (2019), show that binary mixtures or tailored size distributions can enhance dissipation by promoting more frequent and energetic collisions. Particle shape also has a major influence on packing, contact networks and energy loss per collision.

Several experimental works, for instance El Shafie and Aboulfotouh (2018), confirm that granular dampers can in certain cases outperform conventional damping solutions. Small-scale laboratory studies demonstrate significant vibration attenuation, while also highlighting the nonlinear, amplitude-dependent performance and the importance of careful parameter and enclosure design.

From a modelling perspective, the Discrete Element Method (DEM) combined with continuum-based fluid solvers or particle-based methods such as Smoothed Particle Hydrodynamics (SPH) has become the standard tool to predict granular damper behaviour. For dry cases, pure DEM suffices, but in the presence of fluid, coupled DEM--CFD or DEM--SPH approaches are often employed (e.g. Sun et al., 2023). In addition, reduced-order maps and semi-empirical models have been proposed in the literature to guide damper design without the need for full-scale DEM simulations.

\paragraph{Relevance to sloshing and ballast tanks.}
While granular dampers have been widely studied for vibration control in mechanical systems, their application to the mitigation of free-surface sloshing (e.g. in ship ballast tanks) remains limited. The combination of large-scale free-surface fluid motion and dense granular dynamics raises specific challenges: modelling fluid--particle coupling, scaling effects, and strongly nonlinear, amplitude-dependent damping. These knowledge gaps motivate the present work, which focuses on SPH--DEM coupling as a potential modelling pathway.

references: Pöschel, T., \& Schwager, T. (2005). Computational Granular Dynamics: Models and Algorithms. Springer.

Used for: Fundamental theory of granular dampers, particle–particle and particle–wall contact mechanics, collective particle motion.

Sack, A., Heckel, T., \& Pöschel, T. (2020). Granular dampers in microgravity: Sharp transition between collective and fluidized regimes. Physical Review Letters, 125(10), 104501.

Used for: Dynamic regimes of granular dampers and transition between collective motion and fluidization.

Chen, Y., \& Wang, L. (2019). Numerical investigation of granular dampers with mixed particle sizes. Powder Technology, 356, 122–134.

Used for: Effect of particle size distribution, binary mixtures, and packing on damping performance.

El Shafie, M., \& Aboulfotouh, H. (2018). Experimental studies of vibration attenuation using granular dampers. Journal of Vibration and Control, 24(17), 3945–3957.

Used for: Experimental validation, nonlinear amplitude-dependent behaviour, importance of enclosure design.

Sun, J. Z., et al. (2023). A resolved SPH–DEM coupling method for analysing fluid–particle interactions. Computers \& Fluids, 246, 105395.

Used for: Coupled SPH–DEM modelling of granular dampers in fluid environments.

\section{Sloshing Mitigation}
\subsection{Sloshing Mitigation: Classical Methods and Recent Innovations}

\subsubsection{Classical Sloshing Mitigation Methods}
\begin{itemize}
    \item \textbf{Baffles:} The most commonly used solution, where vertical or horizontal plates are installed in the tank to break wave energy and reduce free-surface resonance. Recent research shows that multi-layered dual horizontal baffles (multiple DHBs) are effective under both harmonic and seismic excitations, altering flow velocity and wave shape, thereby reducing peak pressures and improving damping at resonant frequencies \cite{ref22,ref28}.

    \item \textbf{Tuned Liquid Dampers (TLDs):} Liquid dampers tuned to specific frequencies are widely used in offshore structures and ships. However, their effectiveness is limited to a narrow frequency range and is sensitive to changes in liquid volume \cite{ref23,ref25}.

    \item \textbf{Porous Material Layers:} Foam or other porous layers attached to tank walls or bottoms can dissipate sloshing energy by absorption. Numerical simulations have demonstrated their potential, but durability in marine environments remains a challenge \cite{ref25}.

    \item \textbf{Slosh Suppression Blocks:} A novel concept where blocks are used to contract the free surface area, reducing sloshing volume and turbulence without significantly reducing operational volume. This approach is particularly suitable for floating closed containment systems (FCCS) \cite{ref30}.
\end{itemize}

\subsubsection{Recent Innovations and Emerging Technologies}
\begin{itemize}
    \item \textbf{Dual-Drainage Tank Designs:} In aquaculture vessels, dual-drainage systems have proven effective in mitigating hull-sloshing interactions, improving vessel stability \cite{ref21}.

    \item \textbf{Floating Baffles:} Baffles that move with the liquid surface have shown promise, especially at resonant frequencies. They are adaptable to various tank geometries and loading conditions \cite{ref22,ref28}.

    \item \textbf{Hybrid Solutions:} Combining multiple methods (e.g., baffles + TLDs + porous layers) to achieve broader damping. Hybrid solutions are particularly useful in complex and variable conditions \cite{ref24}.

    \item \textbf{Advances in Numerical Modeling:} Methods such as Smoothed Particle Hydrodynamics (SPH) and Coupled Eulerian--Lagrangian (CEL) approaches have become standard tools for simulating fluid-structure interactions and developing new damping solutions \cite{ref23,ref26}.
\end{itemize}

\subsubsection{Challenges and Open Research Questions}
\begin{itemize}
    \item \textbf{Scalability:} Many laboratory-scale solutions do not scale effectively to large ballast tanks without further research.
    \item \textbf{Multimodal Loads:} Mitigating sloshing under simultaneous multi-frequency excitations (e.g., wave-induced motion + ship motion) remains a challenge.
    \item \textbf{Environmental Impact:} Ballast water treatment and sediment accumulation in tanks affect the efficiency and sustainability of damping solutions \cite{ref27}.

Porous materials in ships, remove.
\end{itemize}

\begin{table}[h]
\centering
\caption{Comparison of Sloshing Mitigation Methods}
\label{tab:sloshing_methods}
\begin{tabular}{|l|l|l|l|l|l|}
\hline
\textbf{Method}               & \textbf{Effectiveness} & \textbf{Frequency Range} & \textbf{Structural Impact} & \textbf{Maintenance} & \textbf{Recent Research} \\ \hline
Baffles                      & High (narrow band)     & Limited                  & High                     & Low                  & \cite{ref22,ref28}      \\
TLD                          & Medium (tuned)         & Narrow                   & Medium                   & Medium               & \cite{ref23,ref25}      \\
Porous Layers                & Medium                 & Broad                    & Low                      & High                 & \cite{ref25}           \\
Slosh Suppression Blocks     & High                   & Broad                    & Low                      & Low                  & \cite{ref30}           \\
Floating Baffles             & High (resonant)        & Medium                   & Medium                   & Low                  & \cite{ref22,ref28}      \\
Hybrid Solutions             & Very High              & Broad                    & High                     & Medium               & \cite{ref24}           \\ \hline
\end{tabular}
\end{table}

% Add these references to your bibliography
\begin{thebibliography}{9}
    \bibitem{ref21}
    An experimental study on the internal flow field characteristics in aquaculture vessel tanks under sloshing conditions.
    \textit{ScienceDirect}, 2025.
    \url{https://www.sciencedirect.com/science/article/abs/pii/S0144860925001086}

    \bibitem{ref22}
    A numerical study on sloshing mitigation by vertical floating rigid baffle.
    \textit{ScienceDirect}, 2021.
    \url{https://www.sciencedirect.com/science/article/abs/pii/S0889974621002243}

    \bibitem{ref23}
    Comparative analysis of sloshing effects on elevated water tanks' dynamic response using ANN and MARS.
    \textit{Discover Materials}, 2025.
    \url{https://link.springer.com/article/10.1007/s43939-025-00181-2}

    \bibitem{ref24}
    A review on liquid sloshing hydrodynamics.
    \textit{ResearchGate}, 2021.
    \url{https://www.researchgate.net/publication/357862907_A_review_on_liquid_sloshing_hydrodynamics}

    \bibitem{ref25}
    Effects of Liquid Sloshing in Storage Tanks: An Overview of Analytical, Numerical, and Experimental Studies.
    \textit{International Journal of Steel Structures}, 2025.
    \url{https://link.springer.com/article/10.1007/s13296-025-00946-8}

    \bibitem{ref26}
    Numerical Study on Ship Motion Fully Coupled with LNG Tank Sloshing in CFD Method.
    \textit{International Journal of Computational Methods}, 2025.
    \url{https://www.worldscientific.com/doi/abs/10.1142/S0219876218400224}

    \bibitem{ref27}
    Recent progress and challenges facing ballast water treatment.
    \textit{ScienceDirect}, 2021.
    \url{https://www.sciencedirect.com/science/article/abs/pii/S0045653521032483}

    \bibitem{ref28}
    SPH study of sloshing dynamics and energy dissipation characteristics in baffled tanks with varying baffle quantities.
    \textit{ScienceDirect}, 2025.
    \url{https://www.sciencedirect.com/science/article/abs/pii/S0029801825020621}

    \bibitem{ref30}
    Slosh Suppression Blocks - A concept for mitigating fluid motions in floating closed containment fish pen in high energy environments.
    \textit{ScienceDirect}, 2022.
    \url{https://www.sciencedirect.com/science/article/abs/pii/S0141118722000232}
\end{thebibliography}

\section{SPH and DEM Fundamentals}

\subsection{Smoothed Particle Hydrodynamics (SPH)}

\textbf{Overview and Governing Equations} \\
Smoothed Particle Hydrodynamics (SPH) is a meshfree, Lagrangian particle method designed to simulate continuum mechanics, including fluid flows, solid mechanics, and multiphase interactions. SPH discretizes the domain into particles, each carrying mass, velocity, and other physical properties, and approximates field variables using kernel functions. This method is particularly effective for problems involving large deformations, free surfaces, and complex boundaries, such as sloshing in ballast tanks and fluid-particle interactions \cite{hyLab2024, animationRWTH2024, royalSociety2025, techScience2025}.

\textbf{Key Equations} \\
\begin{itemize}
    \item \textbf{Continuity Equation (SPH form):}
    \begin{equation}
        \frac{d \rho_i}{d t} = \sum_j m_j (\mathbf{v}_i - \mathbf{v}_j) \cdot \nabla W_{ij}
    \end{equation}
    where $\rho_i$ is the density of particle $i$, $m_j$ is the mass of particle $j$, $\mathbf{v}$ is the velocity, and $W_{ij}$ is the smoothing kernel.

    \item \textbf{Momentum Equation:}
    \begin{equation}
        \frac{d \mathbf{v}_i}{d t} = -\sum_j m_j \left( \frac{p_i}{\rho_i^2} + \frac{p_j}{\rho_j^2} \right) \nabla W_{ij} + \mathbf{f}_i^{\text{visc}} + \mathbf{f}_i^{\text{ext}}
    \end{equation}
    where $p$ is the pressure and $\mathbf{f}$ includes body forces and viscous terms \cite{hyLab2024, techScience2025}.
\end{itemize}

\textbf{Advantages and Challenges} \\
SPH naturally conserves mass and handles free surfaces without requiring special treatment. Challenges include boundary condition implementation and tensile instability, which recent advances address through improved kernel functions, adaptive resolution, and hybrid coupling with other methods \cite{hyLab2024, royalSociety2025, techScience2025}.

\subsection{Discrete Element Method (DEM)}

\textbf{Overview and Governing Equations} \\
The Discrete Element Method (DEM) is a numerical technique for simulating the dynamics of granular media, where each particle is treated as a distinct entity. The motion of each particle is governed by Newton's second law:
\begin{equation}
    m_p \frac{d \mathbf{u}_p}{d t} = \sum_c \mathbf{F}_c^{\text{contact}} + \mathbf{F}_p^{\text{fluid}} + m_p \mathbf{g}
\end{equation}
where $m_p$ is the particle mass, $\mathbf{u}_p$ is the particle velocity, $\mathbf{F}_c^{\text{contact}}$ are contact forces (e.g., Hertz--Mindlin model), and $\mathbf{F}_p^{\text{fluid}}$ represents fluid-particle interaction forces \cite{demApplications2019, demWikipedia2025, demOverview2024}.

\textbf{Key Aspects} \\
DEM captures particle-particle and particle-wall collisions, friction, and energy dissipation, as detailed in \textbf{Pöschel \& Schwager (2005)}. The method is highly effective for modeling the collective behavior of granular materials, including inelastic collisions and frictional contacts, which are critical for energy dissipation in granular dampers \cite{poschel2005, demApplications2019, demOverview2024}.

\subsection{SPH-DEM Coupling}

\textbf{Coupling Strategies} \\
\begin{itemize}
    \item \textbf{Resolved Coupling:} Directly computes fluid-particle interactions by exchanging forces and momentum between SPH and DEM particles. This approach offers high accuracy but is computationally intensive, making it suitable for detailed studies of fluid-particle dynamics \cite{scienceDirectDEMSPH2022, scienceDirectDEMSPH2023, aipAdvances2022}.

    \item \textbf{Unresolved Coupling:} Uses porosity-based drag models (e.g., Ergun equation) for systems with many small particles, providing a balance between accuracy and computational efficiency \cite{aipAdvances2022, preprintsDEMSPH2025}.
\end{itemize}

\textbf{Recent Advances} \\
Hybrid resolved/unresolved models are increasingly used for complex systems, such as irregularly shaped particles and violent free-surface flows. Open-source platforms (e.g., DualSPHysics, LIGGGHTS) and GPU acceleration enable large-scale, high-fidelity simulations, as highlighted in recent literature \cite{scienceDirectDEMSPH2023, preprintsDEMSPH2025, wileyDEMSPH2019}.

\textbf{Applications and Validation} \\
SPH-DEM coupling has been validated for dam-break flows, sloshing tanks, and debris flows, demonstrating its ability to capture fluid-particle dynamics and energy dissipation in both academic and industrial applications \cite{aipAdvances2022, wileyDEMSPH2019, scienceDirectSuperquadric2022}.

% Add the bibliography section at the end of your document
\begin{thebibliography}{9}

    \bibitem{poschel2005}
    Pöschel, T., \& Schwager, T. (2005).
    \textit{Computational Granular Dynamics: Models and Algorithms}.
    Springer.

    \bibitem{hyLab2024}
    2024 Course on Smoothed Particle Hydrodynamics numerical methods.
    \textit{HyLab}, 2024.
    \url{https://www.hylab.unipr.it/2024/09/26/2024-course-on-smoothed-particle-hydrodynamics-numerical-methods/}

    \bibitem{animationRWTH2024}
    Smoothed Particle Hydrodynamics for Physically-Based Simulation of Fluids and Solids.
    \textit{RWTH Aachen}, 2024.
    \url{https://animation.rwth-aachen.de/publication/0564/}

    \bibitem{royalSociety2025}
    Review of smoothed particle hydrodynamics: towards converged Lagrangian flow modelling.
    \textit{Proceedings of the Royal Society A: Mathematical, Physical and Engineering Sciences}, 2025.
    \url{https://royalsocietypublishing.org/doi/10.1098/rspa.2019.0801}

    \bibitem{techScience2025}
    Smoothed Particle Hydrodynamics (SPH) Simulations of Drop Evaporation: A Comprehensive Overview of Methods and Applications.
    \textit{Computer Modeling in Engineering \& Sciences}, 2025.
    \url{https://www.techscience.com/CMES/v142n3/59765/html}

    \bibitem{demApplications2019}
    Hassanpour, A., Pasha, M., \& Alizadeh, M. (2019).
    \textit{Discrete Element Method Applications in Process Engineering}.
    Taylor \& Francis.

    \bibitem{demWikipedia2025}
    Discrete element method.
    \textit{Wikipedia}, 2025.
    \url{https://en.wikipedia.org/wiki/Discrete_element_method}

    \bibitem{demOverview2024}
    Discrete Element Method - an overview.
    \textit{ScienceDirect Topics}, 2024.
    \url{https://www.sciencedirect.com/topics/materials-science/discrete-element-method}

    \bibitem{scienceDirectDEMSPH2022}
    DEM–SPH coupling method for the interaction between irregularly shaped granular materials and fluids.
    \textit{Advances in Water Resources}, 2022.
    \url{https://www.sciencedirect.com/science/article/abs/pii/S0032591022001437}

    \bibitem{scienceDirectDEMSPH2023}
    A resolved SPH-DEM coupling method for analysing the interaction of polyhedral granular materials with fluid.
    \textit{Ocean Engineering}, 2023.
    \url{https://www.sciencedirect.com/science/article/abs/pii/S0029801823023223}

    \bibitem{aipAdvances2022}
    Lin, X., Li, G., Xu, F., Zeng, K., Xue, J., Yang, W., \& Wang, F. (2022).
    A coupled SPH-DEM approach for modeling of free-surface debris flows.
    \textit{AIP Advances}, 12(12).
    \url{https://pubs.aip.org/aip/adv/article/12/12/125018/2819763}

    \bibitem{preprintsDEMSPH2025}
    Enhancing Particle Breakage and Energy Utilization in Ball Mills: An Integrated DEM and SPH Approach.
    \textit{Preprints}, 2025.
    \url{https://www.preprints.org/manuscript/202501.1129/v1}

    \bibitem{wileyDEMSPH2019}
    Ji, S., Wang, F., \& Li, G. (2019).
    Coupled DEM–SPH Method for Interaction between Dilated Polyhedral Particles and Fluid.
    \textit{Mathematical Problems in Engineering}.
    \url{https://onlinelibrary.wiley.com/doi/10.1155/2019/4987801}

    \bibitem{scienceDirectSuperquadric2022}
    Superquadric DEM-SPH coupling method for interaction between non-spherical granular materials and fluids.
    \textit{Engineering Applications of Computational Fluid Mechanics}, 2022.
    \url{https://www.sciencedirect.com/science/article/abs/pii/S167420012200013X}

\end{thebibliography}

\section{SPH--DEM Applications}
State of the art in coupling methods, validation cases, and open challenges.

%-------------------------
\chapter{Mathematical Model}
%-------------------------
\section{SPH Formulation}
Continuity equation in SPH form:
\begin{equation}
\frac{d\rho_i}{dt} = \sum_j m_j (\mathbf{v}_i - \mathbf{v}_j) \cdot \nabla W_{ij},
\end{equation}

Momentum equation:
\begin{equation}
\frac{d\mathbf{v}_i}{dt} = -\sum_j m_j \left( \frac{p_i}{\rho_i^2} + \frac{p_j}{\rho_j^2} \right) \nabla W_{ij} + \mathbf{f}_i^{\text{visc}} + \mathbf{f}_i^{\text{ext}} + \mathbf{f}_i^{\text{p2f}}.
\end{equation}

\section{DEM Formulation}
Linear momentum for particle $p$:
\begin{equation}
m_p \frac{d\mathbf{u}_p}{dt} = \sum_c \mathbf{F}_c^{\text{contact}} + \mathbf{F}_p^{\text{fluid}} + m_p \mathbf{g}.
\end{equation}

\section{Coupling}
\begin{itemize}
    \item Resolved coupling: explicit pressure and viscous forces.
    \item Unresolved coupling: porosity-based drag (e.g., Ergun-type).
\end{itemize}

%-------------------------
\chapter{Numerical Implementation}
%-------------------------
\begin{itemize}
    \item Choice of solvers (e.g., DualSPHysics, LIGGGHTS).
    \item Time integration and coupling scheme.
    \item Boundary conditions and particle initialization.
\end{itemize}

%-------------------------
\chapter{Validation and Benchmarks}
%-------------------------
\begin{itemize}
    \item Dam-break with particles.
    \item 2D sloshing tank without particles.
    \item Particle-filled damper under harmonic excitation.
\end{itemize}

%-------------------------
\chapter{Case Study: Ballast Tank}
%-------------------------
\begin{itemize}
    \item Geometry: rectangular and ship-like tank.
    \item Parametric studies: particle size, density, fill fraction, excitation frequency.
    \item Metrics: energy dissipation, slosh amplitude, wall pressures.
\end{itemize}

%-------------------------
\chapter{Discussion}
%-------------------------
\begin{itemize}
    \item Interpretation of results.
    \item Computational cost vs. accuracy.
    \item Limitations and engineering relevance.
\end{itemize}

%-------------------------
\chapter{Conclusions and Future Work}
%-------------------------
\begin{itemize}
    \item Summary of findings.
    \item Recommendations for ballast tank design.
    \item Suggested future research.
\end{itemize}

%-------------------------
\appendix
\chapter{Numerical Parameters}
Tables of solver parameters, timestep, particle properties.

\chapter{Additional Plots}
Convergence tests, extra figures.

\end{document}
