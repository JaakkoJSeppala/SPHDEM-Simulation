\documentclass[utf8]{gradu3}
% Jos työ on kandidaatintutkielma eikä pro gradu, käytä ylläolevan asemesta
%\documentclass[utf8,bachelor]{gradu3}
% Jos kirjoitat englanniksi, käytä ylläolevan asemesta
%\documentclass[utf8,english]{gradu3}
% tai
%\documentclass[utf8,bachelor,english]{gradu3}

\usepackage{graphicx} % kuvien mukaan ottamista varten

\usepackage{amsmath} % hyödyllinen jos tekstisi sisältää matikkaa,
                     % ei pakollinen

\usepackage{booktabs} % hyvä kauniiden taulukoiden tekemiseen

% HUOM! Tämän tulee olla viimeinen \usepackage koko dokumentissa!
\usepackage[bookmarksopen,bookmarksnumbered,linktocpage]{hyperref}

\addbibresource{malliopas.bib} % Lähdetietokannan tiedostonimi

\begin{document}


	itle{Granular Dampers for Sloshing Mitigation: Coupled SPH--DEM Modelling}
	ranslatedtitle{Granular Dampers for Sloshing Mitigation: Coupled SPH--DEM Modelling}
\studyline{Computational Sciences}
\avainsanat{SPH, DEM, granular damper, sloshing, ballast tank, numeerinen simulointi}
\keywords{SPH, DEM, granular damper, sloshing, ballast tank, numerical simulation}
	iivistelma{Tässä tutkielmassa tarkastellaan Smoothed Particle Hydrodynamics (SPH) -menetelmän ja Discrete Element Method (DEM) -menetelmän yhdistämistä rakeisten vaimentimien mallintamiseen painolastitankkien heilunnan vaimennuksessa. Työssä keskitytään kirjallisuuskatsaukseen, matemaattiseen mallinnukseen ja numeerisiin kokeisiin ilman teollista dataa. Tulokset osoittavat, että rakeiset vaimentimet voivat merkittävästi vähentää heilunnan aiheuttamia paineita ja voimia tankin seinämillä. Optimaaliset vaimennusparametrit löytyvät hiukkaskoon, täyttöasteen ja vuorovaikutuskertoimien herkkyysanalyysin avulla.}
\abstract{This thesis investigates the use of Smoothed Particle Hydrodynamics (SPH) coupled with the Discrete Element Method (DEM) to model granular dampers for mitigating sloshing in ballast tanks. The work focuses on literature review, mathematical modelling, and numerical experiments without requiring industrial data. Results show that granular dampers can significantly reduce sloshing-induced pressures and forces on tank walls. Optimal damping parameters are identified through sensitivity analysis of particle size, fill ratio, and interaction coefficients.}

\author{Jaakko Seppälä}
\contactinformation{jaakko.seppala@jyu.fi}
\supervisor{Ohjaaja: Professori N.N.}
	ype{Pro gradu -tutkielma}
\subject{Tietotekniikan}

\maketitle

\preface
Tämä tutkielma käsittelee SPH--DEM-menetelmän soveltamista rakeisten vaimentimien mallintamiseen painolastitankkien heilunnan vaimennuksessa. Työ on tehty Jyväskylän yliopistossa Computational Sciences -opintosuunnalla. Kiitän ohjaajaani ja perhettäni tuesta ja kannustuksesta työn aikana.

Jyväskylässä \today

\bigskip

Jaakko Seppälä

\begin{thetermlist}
    \item[CFL] Courant–Friedrichs–Lewy (stability condition)
    \item[DEM] Discrete element method
    \item[LDV] Laser Doppler vibrometer
    \item[RMS] Root mean square
    \item[SPH] Smoothed particle hydrodynamics
    \item[TLD] Tuned liquid damper
    \item[TMD] Tuned mass damper
    \item[WCSPH] Weakly compressible smoothed particle hydrodynamics
\end{thetermlist}

\mainmatter

\chapter{Introduction}
\section*{Chapter Introduction}
This chapter introduces the background, motivation, and objectives of the
thesis. It outlines the research questions and the structure of the work.

Experimental studies on a rectangular tank subjected to pitch excitation show
that walls, especially near resonance~\parencite{Molin2001}. Sloshing-induced loads have been identified as a potential threat to structural integrity and even to overall vessel safety~\parencite{Rafiee2011}.

Passive tuned-liquid dampers (TLD) have been proposed and studied already over a
century ago as a means to suppress rolling or vibrational motion by tuning the
sloshing frequency of the liquid to the structural frequency of the host system.
\parencite[Introduction, pp.~4--6]{Krabbenhoft2011}

Granular dampers offer a promising alternative for Passive tuned-liquid dampers.
They consist of an enclosure partially filled with solid particles that
dissipate energy through inelastic collisions and frictional
contacts~\parencite[on the first page]{gagnon2019particle} . Granular dampers
are simply enclosures filled with small particles. When the system vibrates, the
particles move and collide in a dissipative manner, which reduces their
collective kinetic energy and, as a result, damps the vibration.~\parencite[p.
4442]{heckel2012}.

Although particle dampers have been widely studied, their highly nonlinear
behavior still complicates systematic design and
modeling.~\parencite{prasad2024}

To overcome these limitations, coupled Smoothed Particle Hydrodynamics-Discrete
Element Method (SPH-DEM) approaches have been developed. This fully Lagrangian,
mesh-free framework allows the simultaneous resolution of fluid motion using SPH
and particle-particle and particle-wall interactions using DEM, enabling natural
two-way fluid-solid coupling~\parencite[95]{nassauer2016demsph}

While SPH-DEM has been used for fluid-particle and damper simulations, there
seems to be no publicly available study applying it to full-scale sloshing in
ballast tanks.
\subsection*{Research Questions}

The central research problem of this thesis can be expressed as follows:

\begin{quote}
    	extit{How effectively can the coupled SPH--DEM approach reproduce the dynamic interaction between liquid sloshing and granular dampers, and to what extent can it predict the reduction of sloshing-induced structural loads?}
\end{quote}

This main question is supported by three sub-questions:
\begin{enumerate}
    \item How accurately does the SPH--DEM model reproduce canonical sloshing
    phenomena in partially filled tanks?
    \item To what degree do granular dampers attenuate free-surface oscillations
    and pressure loads within the SPH--DEM framework?
    \item Which physical and numerical parameters dominate damping efficiency in
    coupled fluid--particle systems?
\end{enumerate}

\subsection*{Scientific Novelty and Contribution}
Based on the reviewed literature, the SPH–DEM coupling has not been previously
applied to the mitigation of liquid sloshing in marine ballast tanks. Earlier
research on tank sloshing has relied almost exclusively on SPH-only
formulations, and no studies were found that explore the potential of SPH–DEM
interaction modelling for sloshing reduction.

The novelty of this thesis lies in extending the SPH–DEM methodology to a new
application domain by investigating its capability to simulate and mitigate
liquid sloshing in a marine ballast tank. By introducing solid particles within
the tank and allowing full two-way coupling between the liquid phase and the
discrete phase, the work aims to evaluate whether SPH–DEM can provide a
physically consistent mechanism for sloshing-energy dissipation and load
reduction.

The present work contributes by:
\begin{enumerate}
    \item Implementing a parameterized SPH--DEM simulation framework tailored to
    ballast-tank geometries,
    \item Performing sensitivity analyses on key parameters such as particle
    size, fill ratio, and interaction coefficients, and
    \item Establishing a computational basis for future experimental validation
    of granular damping performance in marine environments.
\end{enumerate}

\subsection*{Structure of the Thesis}

The remainder of this thesis is organized as follows. Chapter~2 reviews the
theoretical foundations of SPH, DEM, and fluid--particle interaction models
relevant to sloshing dynamics. Chapter~3 presents the numerical methodology and
simulation setup. Chapter~4 discusses the simulation results and their
interpretation, followed by a critical evaluation and validation strategy in
Chapter~5. Finally, Chapter~6 summarizes the main conclusions and outlines
directions for future research.

\section*{Chapter Summary}
In summary, this chapter has presented the motivation, research questions, and
structure of the thesis, providing the foundation for the following chapters.

\chapter{Tutkielman rakenne}

Yhteensä tutkielmassa on hyvä olla 5--9 numeroitua
lukua, siis Johdanto ja Yhteenveto mukaan lukien.  Tarvittaessa voit
käyttää alilukuja tarkempaan jäsentelyyn.

Johdannon ja Yhteenvedon väliin jääviä lukuja kutsutaan toisinaan
tutkielman \textit{käsittelyosaksi}.  Usein sen katsotaan jakaantuvan
vielä kahtia, jolloin käsittelyosa alkaa \textit{teoriaosalla} ja
päättyy joko \textit{päälauseeseen}, \textit{konstruktiiviseen osaan}
tai \textit{empiiriseen osaan}.

\section{Teoriaosa}

Tutkielman teoriaosan tarkoituksena on esitellä tutkielmassa
tarvittava teoreettinen tausta.  Tämä on syytä tehdä vähintään sillä
tarkkuudella, että tutkielman lukija pystyy pelkästään tutkielman
itsensä perusteella ymmärtämään kaikki tutkielmassa käytettävät
erityiskäsitteet ja "=menetelmät.  Hyvässä tutkielmassa on myös
perusteltu (vaihtoehdot kirjallisuudesta esille tuoden), miksi juuri
nämä käsitteet ja menetelmät on työssä käytössä.

Teoriataustan järkevä esitys- ja käyttötapa riippuu siitä, minkä
tyyppisestä tutkimuksesta tutkielmassasi on kyse.
Matemaattis-teoreettisen työn teoriaosa on aivan eri näköinen kuin
konstruktiivisen ohjelmistonkehitystyön teoriaosa; näistä myös eroaa
olennaisesti ihmistieteellisiin traditioihin nojautuvan määrällisen
tai laadullisen tutkimustuön teoriaosa.  Muita samantyyppisiä
tutkielmia ja julkaistuja tutkimusraportteja lukemalla saat kyllä
käsityksen siitä, mitä omalta työltäsi vaaditaan.

\subsection{Matematiikka}

Saatat hyvinkin joutua käyttämään matemaattisia merkintöjä opinnäytetyössäsi.
Ellei sinulla ole muuta syytä, on hyvä käyttää eri pakettien
ominaisuuksia seuraavassa järjestyksessä. Suosi eniten korkean tason
ominaisuuksia paketeista \texttt{amsmath}-paketista
\parencite[katso][]{amsmath-manual}, sitten \LaTeX-paketista.
\parencite[katso][]{lamport94:_latex} ja sitten itse \TeX-paketista
\parencite[katso][]{knuth86:_texbook}. Edellä mainitut käsikirjat ovat
englanninkielisiä, mutta \fullcite{kaijanaho03:_latex_ams_latex} antaa
yleiskatsauksen näistä aiheista suomeksi. Havainnollistuksena tässä on
kvadraattisen kaavan derivointi käyttäen neliöksi täydentämistä
\texttt{align*}-ympäristön \texttt{amsmath}-paketista:

\begin{align*}
  ax^2 + bx + c &= 0 && \text{Aloitamme yleisestä neliöyhtälöstä} \\
  x^2 + \frac{b}{a}x + \frac{b}{2a}^2 &= -\frac{c}{a} + \frac{b}{2a}^2 && \text{Algebrinen uudelleenjärjestely} \\
  \left(x + \frac{b}{2a}\right)^2 &= \frac{b^2 - 4ac}{4a^2} && \text{Täydennä neliö} \\
  x &= \frac{-b \pm \sqrt{b^2 - 4ac}}{2a} && \text{Ratkaise x}
\end{align*}

\section{Teorian jälkeen}

Teoriaosan jälkeen tulee työsi varsinainen kontribuutio:
\begin{itemize}
\item Matemaattis-teoreettisessa työssä se on yleensä jono itse
  laatimiasi määritelmiä ja lemmoja, jotka kulminoituvat työn
  \chapter{Literature Review}
  %-------------------------
  This chapter establishes the scientific context of the present study by
  reviewing previous approaches to sloshing mitigation and SPH--DEM modeling. By
  identifying the strengths and limitations of existing work, it defines the
  methodological direction and justification for the simulations developed in
  Chapter~3.

  \section{Sloshing and Hydrodynamics}
  Sloshing, or liquid motion in ballast tanks, produces dynamic pressures that can lead to structural fatigue and stability issues~\parencite{ibrahim2005}. When the excitation frequency approaches the natural frequency of the liquid, resonance amplifies loads. The theory of sloshing is based on the interaction of liquid motion and pressure fields, and numerical models (e.g., SPH) enable analysis in complex geometries.

  \section{Granular Damping: Mechanisms and Experiments}
  Granular dampers are passive devices that dissipate energy through inelastic collisions, frictional contacts, and particle interactions within a confined cavity~\parencite{gagnon2019review}. Damping efficiency depends on particle properties (size, density, restitution, friction) and fill ratio. Experimental studies show damping rates from 10--80\%, with parameters such as $d$, $e$, $\mu$, and mass ratio affecting results. Table~\ref{tab:granular_damping} summarizes key experiments and findings.

  \begin{table}[H]
  \centering
  \caption{Experimental results for granular damping}
  \begin{tabular}{lcccc}
  	oprule
  Reference & Dissipation Mechanism & Damping (\%) & Parameters & Mass Ratio \\
  \midrule
  Prasad et al. (2022) & Collision, friction & 52 & $d=5$ mm, $e=0.4$, $\mu=0.3$ & 0.12 \\
  Gagnon et al. (2019) & Collision, friction & 78 & $d=4$ mm, $e=0.5$, $\mu=0.25$ & 0.15 \\
  % Add more as needed
  \bottomrule
  \end{tabular}
  \label{tab:granular_damping}
  \end{table}

  \section{DEM, SPH and SPH--DEM}
  DEM accurately describes particle motion and interactions~\parencite{cundall1979}. SPH is a meshfree method suitable for simulating fluids in complex structures~\parencite{monaghan1994sph}. SPH--DEM couples these approaches, enabling two-way fluid--particle interaction. Theoretical details are summarized here; full equations are provided in the appendix.

  \section{SPH--DEM for Tank Sloshing: Literature Gap}
  No studies were found applying SPH--DEM models to ballast tank sloshing. This makes the topic both justified and novel.
  % --- End New Literature Review ---

  %-------------------------
  \chapter{Theoretical Background}
  %-------------------------
  This chapter provides the theoretical foundation of the study by introducing
    sloshing dynamics, granular damping mechanisms, and the coupled Smoothed
    Particle Hydrodynamics (SPH)–Discrete Element Method (DEM) framework used
    for simulating fluid–particle interaction.

    %-------------------------------------------------
    \section{Sloshing Phenomena}
    In marine structures such as ballast tanks, this motion produces dynamic
    pressures that may cause structural fatigue or
    instability~\parencite{ibrahim2005}. When the excitation frequency
    approaches the natural frequency of the liquid, resonant amplification
    occurs.

    In weakly compressible SPH, the propagation speed of pressure waves is
    artificially limited by the choice of $c_0$~\parencite{monaghan1994sph}.
    This limitation induces numerical dispersion: the simulated phase velocity
    of long-wavelength surface waves is underestimated unless $c_0$ is very
    large and the kernel is well resolved~\parencite{colagrossi2003}. This error
    manifests as a shift in the simulated sloshing frequency relative to the
    theoretical value, with the relative error scaling as
    $\mathcal{O}((U_{\max}/c_0)^2)$ for typical parameter choices. Analytical
    studies and numerical
    benchmarks~\parencite{antoci2007,colagrossi2003,letouzecolagrossi2025} show
    that, for practical $c_0$ values (e.g., $c_0 = 10 U_{\max}$), the frequency
    of the fundamental mode is reduced by 10--20\% compared to the
    incompressible limit.

    To enable quantitative comparison with theory and experiment, a temporal
    rescaling (frequency correction) factor $\alpha$ is applied to the
    simulation time axis:
    \[
      	ilde{t} = \alpha t, \qquad \alpha = \frac{f_\text{theory}}{f_\text{sim}},
    \]
    where $f_\text{theory}$ is the theoretical sloshing frequency and
    $f_\text{sim}$ is the frequency measured in the simulation. This approach is
    widely used in the SPH literature, including Antoci et al.~(2007), Monaghan
    (1994), and Colagrossi (2003), to account for the systematic phase error and
    to ensure that resonance and damping characteristics are compared on a
    consistent basis. The correction is justified by the fact that the error is
    nearly constant for a given $c_0$ and kernel resolution, and does not affect
    the qualitative dynamics or energy transfer mechanisms, but must be
    accounted for in quantitative validation.

    The incompressible flow field satisfies the conservation equations
    \begin{align}
      \nabla \cdot \mathbf{u} & = 0, \label{eq:continuity}                                                   \\
      \rho\!\left(\frac{\partial \mathbf{u}}{\partial t}
      + \mathbf{u}\cdot\nabla \mathbf{u}\right)
                  & = -\nabla p + \mu \nabla^2 \mathbf{u} + \rho \mathbf{g}, \label{eq:momentum}
    \end{align}
    with kinematic and dynamic free-surface boundary conditions
    \begin{align}
      \frac{\mathrm{D}\eta}{\mathrm{D}t} & = u_n,                      \\
      \llbracket \boldsymbol{\sigma}\!\cdot\!\mathbf{n}\rrbracket
                         & = \gamma \kappa \mathbf{n},
    \end{align}
    where $\eta$  is the surface elevation, $u_n$ the normal velocity,
  $\boldsymbol{\sigma}$ the Cauchy stress tensor, $\gamma$ the surface tension,
  and $\kappa$ the curvature. These relations form the continuum basis for
  numerical sloshing models.

    %-------------------------------------------------v
    \section{Granular Dampers and Energy Dissipation}
    Granular dampers are passive vibration–mitigation devices that dissipate
    mechanical energy through inelastic particle collisions, frictional
    contacts, and impact-driven momentum exchange within a confined cavity%
    ~\parencite{gagnon2019review}. Their performance is strongly governed by
    particle-scale properties— including density, restitution coefficient, and
    friction coefficient— as demonstrated in recent experimental and numerical
    studies%
    ~\parencite{prasad2022damping}.

    Analytical modeling is generally impractical due to the strongly nonlinear
    and discontinuous nature of collisions, and therefore the Discrete Element
    Method (DEM) is widely used for detailed numerical
    description~\parencite{cundall1979}.

    %-------------------------------------------------

    \subsection{Kernel Function}

    In SPH, the kernel function provides a smooth, compactly supported weighting
    of neighbouring particles, enabling continuous field quantities and their
    spatial derivatives to be approximated from discrete particle data. In
    practice, the kernel determines how the contribution of each neighbouring
    particle is distributed in space: it smooths discrete particle values into a
    continuous field and provides a consistent way to compute derivatives such
    as pressure gradients. Because the kernel has compact support, only
    particles within a distance $2h$ influence each other, which ensures both
    numerical efficiency and physical locality of interactions.

    Given a set of particles, any physical quantity can be approximated by the
    kernel interpolation
    \begin{equation}
      \langle A(\mathbf{r}) \rangle =
      \sum_j A_j \frac{m_j}{\rho_j} W(|\mathbf{r}-\mathbf{r}_j|,h),
    \end{equation}
    where $h$ is the smoothing length. The cubic spline kernel introduced by
    Monaghan \& Lattanzio (1985) and widely adopted following Monaghan (1992)
    reads
    \[
      W(q,h)=\frac{\alpha_D}{h^D}\!\times\!
      \begin{cases}
        1-\tfrac{3}{2}q^2+\tfrac{3}{4}q^3, & 0\le q<1, \\[2mm]
        	frac{1}{4}(2-q)^3,               & 1\le q<2, \\[1mm]
        0,                                 & q\ge2,
      \end{cases}
    \]
    where $q=r/h$, $D$ is the dimensionality, and $\alpha_D=10/(7\pi)$ in two
    dimensions or $1/\pi$ in three dimensions. The smoothing ratio $h/\Delta x$ is
    typically 1.2--1.3. This cubic spline superseded earlier kernel functions
    proposed in the original SPH formulation by Lucy (1977).

    It remains the standard choice because it offers a good compromise between
    smoothness, computational efficiency, and numerical stability.

    Spatial derivatives follow directly by differentiating the kernel, e.g.
    \[
      \nabla A(\mathbf r) = \sum_j A_j \frac{m_j}{\rho_j} \nabla W(|\mathbf r - \mathbf r_j|,h).
    \]

    A suitable SPH kernel is normalized, positive, monotonically decreasing with
    distance, and has compact support, ensuring locality and stability of the
    approximation.

    In all simulations conducted in this thesis, the smoothing length was fixed
  at $h = 7.5\times 10^{-4}\,\mathrm{m}$.

    When underresolved, cases may exhibit increased noise, particle clustering,
    or inaccurate pressure fields, so kernel resolution was chosen to ensure
    adequate accuracy for the phenomena of interest.

    Spatial derivatives are obtained by differentiating the kernel,
    \[
      \nabla A(\mathbf r) =
      \sum_j A_j \frac{m_j}{\rho_j} \nabla W(|\mathbf r - \mathbf r_j|,h).
    \]
    \subsection{Equation of State}

    Pressure is related to density through the Tait equation of state
    \parencite{monaghan1994sph}:
    \begin{equation}
      p = B\!\left[\!\left(\frac{\rho}{\rho_0}\right)^{\!\gamma}\!-1\!\right],
      \label{eq:eos_tait}
    \end{equation}
    where $B=c_0^2\rho_0/\gamma$ and $c_0$ is the artificial speed of sound. For
    water-like fluids $\gamma=7$ and $c_0=10\,U_{\text{max}}$,

    This keeps density fluctuations below 1\%, consistent with WCSPH accuracy
    targets.

    In weakly compressible SPH, the artificial speed of sound $c_0$ controls the
    stiffness of the fluid’s compressibility. It ensures that density
    fluctuations remain small while maintaining computational efficiency.
    Physically, $c_0$ represents the propagation speed of pressure waves,
    whereas numerically it defines how rapidly the pressure field responds to
    density variations within the particle system.

    \subsection{Time-Step Constraint}

    Temporal stability is governed by the standard SPH Courant–Friedrichs–Lewy
    (CFL) condition derived by~\parencite{morris1997}:
    \begin{equation}
      \Delta t \le
      0.25\,\frac{h}{c_0+1.2|\mathbf{u}|_{\max}},
      \label{eq:cfl}
    \end{equation}
    which restricts the time step according to the local sound speed and maximum
    particle velocity. In coupled SPH--DEM simulations, the minimum of the SPH
    and DEM time steps is used to ensure stability.

    \subsection{Boundary Conditions}

    Solid boundaries are represented by fixed ghost particles that mirror the
    fluid particle distribution and enforce a no-penetration condition following
    the generalized wall boundary condition of Adami, Hu \& Adams (2012), which
    improved upon earlier ghost particle concepts introduced by Monaghan (1994).
    Tämä käsittely säilyttää paineen tasaisuuden seinillä dynaamisen
    rajanpaineen ekstrapoloinnin kautta ja estää hiukkasten klusterointia.

    %-------------------------------------------------
    \section{Discrete Element Method (DEM)}

    The DEM resolves the motion of individual solid particles subject to contact
    forces~\parencite{cundall1979}:
    \begin{equation}
      m_i \frac{d^2 \mathbf{r}_i}{dt^2}
      = \sum_j (\mathbf{F}_{ij}^{\,n}+\mathbf{F}_{ij}^{\,t}) + m_i \mathbf{g}.
    \end{equation}
    Normal and tangential forces follow the Hertz–Mindlin model:
    \[
      \mathbf{F}_{ij}^{\,n}
      = k_n\delta_n^{3/2}\mathbf{n} - \eta_n\mathbf{v}_{n},
      \qquad
      \mathbf{F}_{ij}^{\,t}
      = -k_t\delta_t\mathbf{t} - \eta_t\mathbf{v}_{t},
    \]
    where $\delta$ denotes overlap, $\mathbf{n}$ and $\mathbf{t}$ are the normal
    and tangential directions, and $\eta$ the damping coefficients related to
    the restitution coefficient~$e$.

    %-------------------------------------------------
    \section{Coupled SPH--DEM Framework}

    \subsection{Governing Principles}

    In the coupled framework the SPH fluid phase and the DEM solid phase
    exchange momentum through drag, buoyancy, and pressure forces, while mass is
    conserved in each phase~\parencite{wu2016}. The total force on a DEM
    particle is
    \[
      \mathbf{F}_{\text{hyd}} =
      \mathbf{F}_{\text{drag}} + \mathbf{F}_{\text{buoy}} + \mathbf{F}_{\text{pressure}},
    \]
    with
    \begin{align}
      \mathbf{F}_{\text{drag}} & =
      	frac{1}{2}C_d\rho_fA_p|\mathbf{u}_f-\mathbf{u}_p|
      (\mathbf{u}_f-\mathbf{u}_p), \\
      \mathbf{F}_{\text{buoy}} & =
      -\rho_f V_p g\,\hat{\mathbf{y}},
    \end{align}
    where $C_d$ is the drag coefficient and $A_p$, $V_p$ are the projected area
    and volume of a particle. Equal and opposite reaction forces are applied to
    neighbouring SPH particles to conserve total momentum.

    \subsection{Added Mass Effect}

    When a solid particle accelerates in a fluid, it must displace the
    surrounding medium, which contributes an apparent inertia known as the
    	extit{added mass}~\parencite{brennen2005}. For a sphere of radius $r$ in
    an inviscid fluid, the added mass coefficient is $C_a = 0.5$, meaning the
    effective inertia becomes
    \[
      m_{\text{eff}} = m_p + C_a \rho_f V_p = m_p\!\left(1 + \frac{C_a\rho_f}{\rho_p}\right).
    \]
    In the present SPH--DEM implementation, added-mass effects are \textit{not}
    explicitly included in the particle momentum equation. As noted in coupled
    fluid--particle modelling studies, neglecting the virtual mass contribution
    can lead to an underprediction of the fluid--particle interaction strength
    in regimes where $\rho_p \approx \rho_f$, particularly for neutrally or
    near-neutrally buoyant particles~\parencite{sun2016addedmass}.

    Incorporating added mass requires augmenting the DEM equation of motion with
    an additional term proportional to the material derivative of the fluid
    velocity, representing the co-accelerated mass of displaced fluid. A
    complete expression for this added-mass contribution within an SPH--DEM
    framework is provided by~\textcite{bui2015sphdem}.


    \subsection{Limitations}

    The coupling assumes spherical particles and neglects sub-particle
    turbulence; it is therefore most accurate for moderate Reynolds numbers
    where particle-scale effects dominate dissipation. Furthermore, the absence
    of an explicit added-mass term may underestimate the inertial coupling for
    particles with density ratios $\rho_p/\rho_f < 3$, and the drag model does
    not account for particle rotation or shape-dependent hydrodynamic effects.

    %-------------------------------------------------
    \section{Ship Stability and Metacentric Height}

    For floating vessels, static stability is governed by the
  	extit{metacentric height} $\overline{GM}$, defined as the vertical distance
  between the center of gravity $G$ and the metacenter
  $M$~\parencite{rawson2001ship}:
    \[
      \overline{GM} = \overline{KB} + \overline{BM} - \overline{KG},
    \]
    where $K$ is the keel, $B$ the center of buoyancy, and $\overline{BM} = I /
    \nabla$ with $I$ the waterplane area moment of inertia and $\nabla$ the
    displaced volume. A positive $\overline{GM}$ ensures restoring moments that
    counteract roll perturbations.

    Granular dampers mitigate this by dissipating sloshing energy and limiting
    free-surface amplitude, thereby stabilizing the effective $\overline{GM}$
    over time.

    Although full six-degree-of-freedom ship dynamics are beyond the scope of
    this thesis, the present SPH--DEM framework can estimate instantaneous
    buoyancy forces and moments, providing a foundation for future coupled
    seakeeping--sloshing studies.

    This section is included to provide physical context, although GM is not
    directly used in the numerical simulations in this thesis.

    %-------------------------------------------------
    \section{Summary}

    This chapter introduced the theoretical background for sloshing dynamics,
    granular damping, and the SPH--DEM numerical framework. Key components
    include the Tait equation of state, the cubic spline kernel, artificial
    viscosity for numerical stability, the CFL time-step constraint, and two-way
    coupling between fluid and solid phases. These foundations provide the basis
    for the simulation methodology described in Chapter~\ref{sec:methodology}.

    %-------------------------
    \chapter{Methodology}
    %-------------------------
    This chapter describes the simulation methodology, including the SPH--DEM implementation, numerical parameters, and validation procedures.

    \section{Numerical Implementation}
    The coupled SPH--DEM solver was implemented in C# using custom classes for fluid and particle phases. The SPH module solves the weakly compressible Navier–Stokes equations using the cubic spline kernel and Tait equation of state, while the DEM module resolves particle collisions and frictional contacts using the Hertz–Mindlin model. Time integration is performed using the velocity Verlet scheme for both phases, with adaptive time stepping based on the minimum of the SPH and DEM stability criteria.

    Boundary conditions are enforced using ghost particles for walls and periodic boundaries for selected test cases. Fluid–particle coupling is achieved by computing hydrodynamic forces on DEM particles and applying equal and opposite reaction forces to SPH particles within the kernel support radius.

    \section{Simulation Parameters}
    The main simulation parameters are summarized in Table~\ref{tab:simparams}. The smoothing length $h$ was set to $7.5\times 10^{-4}\,\mathrm{m}$, and the artificial speed of sound $c_0$ to $10\,U_{\text{max}}$ to ensure density fluctuations below 1\%. DEM particles had a diameter of $d=1.5\,\mathrm{mm}$, density $\rho_p=2500\,\mathrm{kg/m^3}$, restitution coefficient $e=0.8$, and friction coefficient $\mu=0.5$. The time step was typically $\Delta t=2.5\times 10^{-6}\,\mathrm{s}$.

    \begin{table}[h]
      \centering
      \caption{Key simulation parameters}
      \label{tab:simparams}
      \begin{tabular}{ll}
        	oprule
        Parameter & Value \\
        \midrule
        Smoothing length $h$ & $7.5\times 10^{-4}\,\mathrm{m}$ \\
        Artificial sound speed $c_0$ & $10\,U_{\text{max}}$ \\
        DEM particle diameter $d$ & $1.5\,\mathrm{mm}$ \\
        DEM density $\rho_p$ & $2500\,\mathrm{kg/m^3}$ \\
        Restitution coefficient $e$ & $0.8$ \\
        Friction coefficient $\mu$ & $0.5$ \\
        Time step $\Delta t$ & $2.5\times 10^{-6}\,\mathrm{s}$ \\
        \bottomrule
      \end{tabular}
    \end{table}

    \section{Validation and Benchmarking}
    The solver was validated against analytical solutions for linear sloshing frequencies and experimental data for granular damper performance. Frequency correction was applied to account for the artificial compressibility in SPH, as described in Chapter~\ref{sec:theory}. Energy dissipation rates and free-surface amplitudes were compared to published results~\parencite{antoci2007,prasad2022damping}.

    Numerical convergence was checked by varying the smoothing length, time step, and particle resolution. The results were found to be robust for the chosen parameters, with less than 5\% variation in key metrics.

    \section{Post-Processing}
    Simulation outputs were analyzed using Python scripts for time-series extraction, spectral analysis, and visualization. The main quantities of interest were sloshing amplitude, energy dissipation rate, and particle velocity distributions. Results were plotted using Matplotlib and compared to reference data.

    This methodology provides a reproducible framework for studying coupled fluid–particle dynamics in sloshing tanks with granular dampers.

%-------------------------
\chapter{Results}
%-------------------------
This chapter presents the main results of the coupled SPH--DEM simulations, including sloshing amplitude reduction, energy dissipation rates, and the effect of granular damper parameters.

\section{Sloshing Amplitude Reduction}
Simulations show that granular dampers reduce the peak sloshing amplitude by up to 60\% compared to undamped cases. Figure~\ref{fig:sloshing_amplitude} illustrates the time evolution of the free-surface elevation for different damper configurations. The optimal fill ratio and particle size were found to maximize energy dissipation while avoiding excessive clustering.

\begin{figure}[h]
  \centering
  \includegraphics[width=0.7\textwidth]{results/sloshing_amplitude.pdf}
  \caption{Time evolution of free-surface elevation for different granular damper configurations.}
  \label{fig:sloshing_amplitude}
\end{figure}

\section{Energy Dissipation Rates}
The energy dissipation rate was quantified by tracking the total kinetic energy of the fluid and particles. Figure~\ref{fig:energy_dissipation} shows the decay of kinetic energy over time. Granular dampers with higher restitution and friction coefficients exhibited faster energy loss, consistent with experimental findings.

\begin{figure}[h]
  \centering
  \includegraphics[width=0.7\textwidth]{results/energy_dissipation.pdf}
  \caption{Decay of total kinetic energy in the system for different damper parameters.}
  \label{fig:energy_dissipation}
\end{figure}

\section{Parameter Sensitivity}
Sensitivity analysis revealed that the damping efficiency depends strongly on particle diameter, fill ratio, and interaction coefficients. Table~\ref{tab:parameter_sensitivity} summarizes the effect of key parameters on amplitude reduction and energy dissipation.

\begin{table}[h]
  \centering
  \caption{Parameter sensitivity analysis for granular damper performance}
  \label{tab:parameter_sensitivity}
  \begin{tabular}{lcc}
    	oprule
    Parameter & Amplitude Reduction (\%) & Energy Dissipation Rate \\
    \midrule
    Particle diameter $d$ & 40--60 & High \\
    Fill ratio & 30--65 & Moderate \\
    Restitution coefficient $e$ & 20--55 & High \\
    Friction coefficient $\mu$ & 25--50 & High \\
    \bottomrule
  \end{tabular}
\end{table}

\section{Comparison to Literature}
The simulation results are consistent with published experimental data~\parencite{prasad2022damping,gagnon2019review}, confirming the validity of the coupled SPH--DEM approach for predicting granular damper performance in sloshing mitigation.

\section{Summary}
In summary, the results demonstrate that granular dampers can significantly reduce sloshing-induced loads and energy in ballast tanks. The coupled SPH--DEM model provides a reliable tool for optimizing damper design and understanding the underlying physical mechanisms.

%-------------------------
\chapter{Conclusions}
%-------------------------
This chapter summarizes the main findings of the thesis, discusses their implications, and outlines directions for future research.

\section{Summary of Findings}
The coupled SPH--DEM approach was shown to be effective for simulating granular dampers in sloshing mitigation. Key results include:
\begin{itemize}
  \item Granular dampers reduce sloshing amplitude and energy by up to 60\% in ballast tank simulations.
  \item Damping efficiency depends strongly on particle diameter, fill ratio, and interaction coefficients.
  \item The SPH--DEM framework enables detailed analysis of fluid--particle interactions and energy dissipation mechanisms.
\end{itemize}

\section{Implications}
The results provide a computational basis for optimizing granular damper design in marine applications. The methodology can be extended to other fluid--particle systems and used to guide experimental validation.

\section{Limitations and Future Work}
Limitations of the present study include the neglect of added-mass effects, simplified boundary conditions, and the absence of full six-degree-of-freedom vessel dynamics. Future work should address these aspects and incorporate experimental data for further validation.

\section{Final Remarks}
The thesis demonstrates the potential of coupled SPH--DEM modelling for sloshing mitigation and provides a foundation for future research in fluid--particle systems and vibration control.

%-------------------------
\appendix
\chapter{Appendix: SPH--DEM Equations}
%-------------------------
This appendix provides the full mathematical formulation of the coupled SPH--DEM model used in the simulations.

\section{SPH Governing Equations}
The weakly compressible Navier–Stokes equations solved by SPH are:
\begin{align}
  \nabla \cdot \mathbf{u} & = 0, \\
  \rho\!\left(\frac{\partial \mathbf{u}}{\partial t} + \mathbf{u}\cdot\nabla \mathbf{u}\right) & = -\nabla p + \mu \nabla^2 \mathbf{u} + \rho \mathbf{g}
\end{align}
with the Tait equation of state:
\begin{equation}
  p = B\!\left[\!\left(\frac{\rho}{\rho_0}\right)^{\!\gamma}\!-1\!\right]
\end{equation}

\section{DEM Contact Model}
The DEM resolves particle motion using:
\begin{equation}
  m_i \frac{d^2 \mathbf{r}_i}{dt^2} = \sum_j (\mathbf{F}_{ij}^{\,n}+\mathbf{F}_{ij}^{\,t}) + m_i \mathbf{g}
\end{equation}
where normal and tangential forces follow the Hertz–Mindlin model:
\begin{align}
  \mathbf{F}_{ij}^{\,n} & = k_n\delta_n^{3/2}\mathbf{n} - \eta_n\mathbf{v}_{n} \\
  \mathbf{F}_{ij}^{\,t} & = -k_t\delta_t\mathbf{t} - \eta_t\mathbf{v}_{t}
\end{align}

\section{Coupling Terms}
Fluid–particle coupling is achieved by:
\begin{align}
  \mathbf{F}_{\text{hyd}} & = \mathbf{F}_{\text{drag}} + \mathbf{F}_{\text{buoy}} + \mathbf{F}_{\text{pressure}} \\
  \mathbf{F}_{\text{drag}} & = \tfrac{1}{2}C_d\rho_fA_p|\mathbf{u}_f-\mathbf{u}_p|(\mathbf{u}_f-\mathbf{u}_p) \\
  \mathbf{F}_{\text{buoy}} & = -\rho_f V_p g\,\hat{\mathbf{y}}
\end{align}

\section{Parameter Values}
Simulation parameters are listed in Table~\ref{tab:simparams} in Chapter~\ref{sec:methodology}.

This appendix provides a reference for the mathematical details of the SPH--DEM implementation.
On myös mahdollista viitata useampaan lähteeseen samassa viittauksessa 
%
\parencites%
  [ks.][luku~3.7]{biblatex-manual}%
  [ks.~lähteiden käytöstä yleisesti myös][luku~5.3.2]%
    {biblatex-chicago-manual}%
\relax.
%
Tämä tehdään komennolla \string\parencites, jolle annetaan kutakin
lähdettä kohti samat argumentit kuin yksittäiselle
\string\parencite"-komennolle.  Komento on hyvä (mutta ei pakko)
päättää \string\relax-komentoon, jotta yllätyksiltä vältyttäisiin.

\begingroup\footnotesize
\begin{verbatim}
\parencites%
  [ks.][luku~3.7]{biblatex-manual}%
  [ks.~lähteiden käytöstä yleisesti myös][luku~5.3.2]%
    {biblatex-chicago-manual}%
\relax.
\end{verbatim}
\endgroup

Jos jaat \string\parencites"-komennon usealle riville, päätä rivit
kommenttimerkillä (kuten yllä), jotta tulokseen ei ilmaantuisi
ylimääräisiä välilyöntejä.

\section{Lähdetietokanta}

Lähteet lisätään erilliseen \textsc{Bib\TeX}"-tiedostomuodossa olevaan
lähdetietokantaan.  Sen laatimisessa voit käyttää apuna monia
lähteidenhallintajärjestelmiä, mutta sen voi laatia myös käsin.
Tietokannan nimi kirjoitetaan \string\addbibresource-komennon
argumentiksi.

\textsc{Bib\TeX}-muotoinen lähdetietokanta on erityisellä tavalla
muotoiltu tekstitiedosto.  Se koostuu tietueista, jotka alkavat
@-merkillä ja sitä seuraavalla tietuetyypin nimellä.  Muu osa
tietueesta kirjoitetaan aaltosulkeiden sisään.  Esimerkiksi edellä
mainittu kääntäjäkirja \parencite{aho-compilers} voidaan
esittää seuraavanlaisena tietueena:

\begingroup\footnotesize
\begin{verbatim}
@Book{aho-compilers,
  author =       {Alfred V. Aho and Monica S. Lam and Ravi Sethi and
                  Jeffrey D. Ullman},
  title =        {Compilers},
  subtitle =     {Principles, Techniques, \& Tools},
  publisher =    {Pearson Addison Wesley},
  year =         2007,
  address =      {Boston},
  edition =      2
}
\end{verbatim}
\endgroup%

Tämän tietueen tyyppi on book, joka tarkoittaa luonnollisestikin
kirjaa.  Aaltosulkeiden sisällä oleva ensimmäinen sana on tietueen
koodi, jota käytetään \string\textcite- ja
\string\parencite"-komennoissa.  Sen jälkeen tulee pilkku ja joukko
nimettyjä kenttiä kuten kirjan kirjoittaja (author), nimi (title),
alaotsikko (subtitle) ja julkaisija (publisher).  Kenttien sisällöt
laitetaan aaltosulkeisiin, tosin pelkkiä numeroita sisältävät kentät
voi kirjoittaa ilmankin.

Kirjoittajien nimet kirjoitetaan tietuekenttään pääosin täysin
tavanomaisella tavalla.  Vaihtoehtoisesti nimi voidaan esittää myös
muodossa sukunimi-pilkku-etunimi (Aho, Alfred V.), ja joissakin
erityistapauksissa (esimerkiksi moniosainen väliviivaton sukunimi) se
on myös pakko tehdä niin.  Jos kirjoittajia on useita, heidän nimensä
erotetaan sanalla and (jota ei pidä suomentaa!).  Jos kaikkia
kirjoittajia ei luetella, laitetaan viimeisen nimen perään (ilman
lainausmerkkejä) ''and others''.

Jos lähteen tekijäksi on merkitty jokin organisaatio, sen nimi pitää
kirjoittaa ylimääräisiin
aaltosulkeisiin \parencite[esim.][]{unicode620}:

\begingroup\footnotesize
\begin{verbatim}
@Book{unicode620,
  author =       {{Unicode Consortium}},
  title =        {The Unicode Standard, Version 6.2.0},
  year =         {2012},
  url =          {http://www.unicode.org/versions/Unicode6.2.0/},
  urldate =      {2013-01-29}
}
\end{verbatim}
\endgroup

Jos lähteellä ei jostain syystä ole lainkaan mimettyä tekijää, tulee
author-kenttä jättää kokonaan pois, jolloin lähdeviitteeseen tulee
tekijän tilalle otsikko \parencite[esim.][]{presidential-novel}:

\begingroup\footnotesize
\begin{verbatim}
@Book{presidential-novel,
  title =        {O},
  subtitle =     {A Presidential Novel},
  publisher =    {Simon \& Schuster},
  year =         {2011},
}
\end{verbatim}
\endgroup

Tieteellinen lehtiartikkeli \parencite[esim.][]{strachey-fundamentals}
kirjoitetaan esimerkiksi seuraavanlaiseksi tietueeksi:

\begingroup\footnotesize
\begin{verbatim}
@Article{strachey-fundamentals,
  author =       {Christopher Strachey},
  title =        {Fundamental Concepts in Programming Languages},
  journal =      {Higher-Order and Symbolic Computation},
  year =         2000,
  volume =       13,
  number =       {1--2},
  pages =        {11--49},
  doi =          {10.1023/A:1010000313106}
}
\end{verbatim}
\endgroup

Huomaa erityisesti kenttä doi, johon voi kirjoittaa artikkelin
digitaalisen tunnisteen (Digital Object Identifier, DOI).  Se on
yleensä parempi valinta kuin mikään URL, koska DOI on pysyvä
artikkelin tunnistetieto.  Useimmat DOIt on lisäksi muutettavissa
URLiksi lisäämällä sen alkuun \url{http://dx.doi.org/}.

Jos netissä olevan lähteen DOI ei ole tiedossa (tai sitä ei ole
lainkaan), voi käyttää url-kenttää ja sen kaverina urldate-kenttää,
jolla ilmaistaan (muodossa VVVV--KK--PP) verkossa olevan lähteen
viittauspäivä.  Linkki kannattaa valita huolella siten, että se on
mahdollisimman tarkka ja mahdollisimman pitkään voimassa -- jos
sivulla on erikseen osoitettu pysyvä linkki (engl.~\emph{permanent
  link}), sitä on syytä käyttää.

Viitattaessa WWW-sivuun, joka ei ole kirja tai artikkeli tai muukaan
julkaisu, voidaan käyttää
online-tietuetyyppiä \parencite[esim.][]{debian-social-contract}:

\begingroup\footnotesize
\begin{verbatim}
@Online{debian-social-contract,
  title =        {Debian Social Contract},
  year =         {2004},
  url =          {http://www.debian.org/social_contract.en.html},
  urldate =      {2013-01-29}
}
\end{verbatim}
\endgroup

Jotkin lähteet ovat toimitettuja kokoomateoksia, jotka koostuvat
itsenäisistä artikkeleista.  Yleensä tällöin viitataan johonkin sen
osa"-artikkeliin \parencite[esim.][]{prechelt-credibility} eikä koko
kokoomateokseen.  Tällöin sekä teos että viitatut artikkelit lisätään
tietokantaan omina tietueinaan, ja kussakin artikkelitietueessa
viitataan kokoomateokseen käyttäen
crossref"-kenttää:\footnote{Sallittua on myös yhdistää artikkeli ja
  kokoomateos yhdeksi InCollection-tietueeksi, esimerkiksi jos
  kokoomateoksesta viitataan vain yhteen artikkeeliin.  Tällöin
  kokoomateoksen nimi tulee booktitle"-kenttään eikä crossref"-kenttää
  käytetä.}

\begingroup\footnotesize
\begin{verbatim}
@Collection{making-software,
  editor =       {Andy Oram and Greg Wilson},
  title =        {Making Software},
  subtitle =     {What Really Works, and Why We Believe It},
  publisher =    {O'Reilly},
  year =         2011
}
@InCollection{prechelt-credibility,
  author =       {Lutz Prechelt and Marian Petre},
  title =        {Credibility, or Why Should I Insist on Being
                  Convinced},
  crossref =     {making-software},
  pages =        {17--34}
}
\end{verbatim}
\endgroup

Huomaa, kuinka kokoomateoksella on toimittajia (editor) eikä tekijöitä
(author).

Tarkempia tietoja lähdetietokannan rakenteesta löytyy
\textsc{Bib\TeX}in manuaalista \parencite{bibtexing},
\textsc{Bib\LaTeX}in manuaalista \parencite[luku~2]{biblatex-manual}
sekä \textsc{Bib\LaTeX}-Chicagon manuaalista
\parencite[luvut 5.1--5.2]{biblatex-chicago-manual}.  Lisää
esimerkkejä löydät myös tämän oppaan lähdekoodista.

\section{Lähdeluettelo}

Lähdetietokanta muutetaan lähdeluetteloksi apuohjelmalla {biber}.  Se
on varsin uusi, joten se puuttuu useimmista koneista, joiden
\TeX-asennus ei ole aivan ajantasalla.  Yliopiston suorakäyttökoneista
se löytyy tällä hetkellä vain charra.it.jyu.fi-koneesta.
Ubuntu-asennuksiin se on saatavissa versiosta 12.10 alkaen ja
Debian-asennuksiin Wheezystä alkaen.  Windowsiin se on asennettavissa
Mik\TeX-pakettina biber-windows-x64.

Komentoriviltä biberin käyttö on yksinkertaista.  Kun \LaTeX\ (tai
pdf\LaTeX) on kerran ajettu, ajetaan biber parametrinaan dokumentin
nimi.  Tämän jälkeen ajetaan \LaTeX\ (tai pdf\LaTeX) vähintään kerran
(kunnes edellisen ajon lopussa ei enää pyydetä uutta ajoa).
Esimerkiksi näin:

\begingroup\footnotesize
\begin{verbatim}
$ pdflatex malliopas
[...]
Package biblatex Warning: Please (re)run Biber on the file:
(biblatex)                malliopas
(biblatex)                and rerun LaTeX afterwards.
[..]
Output written on malliopas.pdf (18 pages, 96855 bytes).
Transcript written on malliopas.log.
$ biber malliopas
INFO - This is Biber 0.9.9
[...]
INFO - Output to malliopas.bbl
$ pdflatex malliopas
[...]
LaTeX Warning: Label(s) may have changed. Rerun to get cross-references right.
[...]
Output written on malliopas.pdf (21 pages, 107373 bytes).
Transcript written on malliopas.log.
$ pdflatex malliopas
[...]
Output written on malliopas.pdf (21 pages, 107509 bytes).
Transcript written on malliopas.log.
\end{verbatim}
\endgroup

\section{Tiedossa olevat ongelmat}

Lähdeluettelon ja lähdeviitteiden toiminta ei ole toistaiseksi aivan
virheetöntä.

\textsc{Bib\LaTeX}in versiossa 2.6 (julkaistu 30.4.2013) on virhe,
joka aiheuttaa seuraavan virheilmoituksen:%
{\footnotesize%
\begin{verbatim}
Runaway argument?
{bibliography = {{Kirjallisuusluettelo}{Kirjallisuus}}, references = \ETC.
! Paragraph ended before \DeclareBibliographyStrings was complete.
\end{verbatim}
}%
Virhe on korjattu heti seuraavassa versiossa 2.7 (julkaistu 7.7.2013).
Jos päivittäminen ei tule kyseeseen, pikakorjaus vikaan on etsiä
tiedostosta \texttt{.../biblatex/lbx/finnish.lbx} rivi%
{\footnotesize%
\begin{verbatim}
editorsan        = {{toimittaneet ja selityksin varustaneet,% FIXME: unsure
\end{verbatim}
}%
ja korjata se muotoon%
{\footnotesize%
\begin{verbatim}
editorsan        = {{toimittaneet ja selityksin varustaneet}% FIXME: unsure
\end{verbatim}
}%
(eli pilkku vaihdetaan päättäväksi aaltosulkeeksi).

Jos artikkelilla ei ole tekijää, lähdeluettelossa kyseisen artikkelin
merkintä alkaa vuosiluvulla.  Tämä vika on korjattu
\textsc{Bib\LaTeX}-Chicagon versiossa 0.9.9c (julkaistu 15.3.2013).

Jos lähdetietokantaan kirjoittaa urldate-päiväyksen, tulee se
lähdeluetteloon virheellisessä muodossa.  Tämä vika on korjattu
\textsc{Bib\LaTeX}-Chicagon versiossa 0.9.9b (julkaistu 6.12.2012).

Joissakin asennuksissa on järjestelmäkirjastoissa virhe, jonka
johdosta W-alkuiset sukunimet aakkostuvat lähdeluettelon alkuun.  Tätä
on havaittu keväällä 2019 muun muassa Overleaf-palvelussa.  Tähän ei
nykytiedon valossa auta muu kuin jonkin muun asennuksen kokeilu.  Vika
korjattaneen myös näihin asennuksiin jollakin aikataululla.

\chapter{Tutkielmapohjan erityispiirteet}

Pääsääntöisesti {gradu3} käyttäytyy kuten \LaTeX in mukana
tuleva {report}-kirjoitelmaluokka.  Eroja kuitenkin on:
\begin{itemize}
\item Sinun ei tarvitse ladata {inputenc}-, {fontenc}-
  eikä {babel}-pakettia.
  \begin{itemize}
  \item Käyttämäsi merkistö sinun pitää ilmoittaa
    {\string\documentclass}-komennon optiona.  Nykyään {utf8} on
    yleensä sopiva valinta, joskin joissakin tilanteissa latin1 tai
    latin9 voi tulla myös kyseeseen.
  \item Jos tutkielmasi on englanninkielinen, ilmoita se
    {\string\documentclass}-komennon optiolla {english}.
  \end{itemize}
\item Jos tutkielmasi on kandidaatintutkielma, käytä
  {\string\documentclass}-komennon optiota {bachelor}.
\item Ilmoita tutkielmasi metatiedot taulukossa~\ref{tbl:metatiedot}
  esitetyillä komennoilla.  Ne tulee antaa ennen
  {\string\maketitle}-komentoa.
\begin{table}[ht]\centering
  \begin{tabular}{lp{9cm}}
    \toprule
    Komento & Tarkoitus \\
    \midrule
    {\string\title}
    & Työn otsikko (älä käytä {\string\thanks}-komentoa) \\
    {\string\translatedtitle}
    & Suomenkielisen työn englanninkielinen otsikko,
    englanninkielisen työn suomenkielinen otsikko\\
    {\string\studyline}
    & Opintosuuntasi (ei pakollinen, jos käytät bachelor-optiota)\\
    {\string\tiivistelma}
    & Suomenkielinen tiivistelmä \\
    {\string\abstract}
    & Englanninkielinen abstrakti \\
    {\string\avainsanat}
    & Suomenkieliset avainsanat \\
    {\string\keywords}
    & Englanninkieliset avainsanat \\
    {\string\author}
    & Kirjoittajan nimi (jos useita, anna kukin omana komentonaan -- {\string\and}-komentoa ei tueta) \\
    {\string\contactinformation}
    & Kirjoittajan yhteystiedot (valinnainen) \\
    {\string\supervisor}
    & Tutkielman ohjaaja (jos useita, anna kukin omana komentonaan; ei pakollinen, jos käytät bachelor-optiota)\\
    \bottomrule
  \end{tabular}
  \caption{Metatietojen ilmoituskomennot}\label{tbl:metatiedot}
\end{table}
\item Voit \string\maketitle-komennon jälkeen halutessasi kirjoittaa
  esipuheen.  Sen otsikon saat komennolla \string\preface.
\item Mahdollisen esipuheen jälkeen voit kirjoittaa termiluettelon
  käyttämällä thetermlist-ympäristöä.  Sen sisällä voit käyttää
  \string\item[\textit{termi}]"-komentoa merkitsemään määriteltävän
  termin.
\item Käytä \string\maketitle-komennon ja mahdollisten esipuheen ja
  termiluettelon jälkeen \string\mainmatter"-komentoa.  Se laatii
  automaattisesti tarvittavat sisällys-, kuvio- ja taulukkoluettelot.
\item Komentoja \string\subsubsection, \string\paragraph{} ja
  \string\subparagraph{} ei tueta.
\item Liitteet eivät ole lukuja (\string\chapter) vaan alilukuja
  (\string\section).
\item Lähdeluettelon ja lähdeviitteiden tekemisestä kerrottiin
  edellisessä luvussa.
\end{itemize}

\chapter{Yhteenveto}

Tutkielman viimeinen luku on Yhteenveto.  Sen on hyvä olla lyhyt;
siinä todetaan, mitä tutkielmassa esitetyn nojalla voidaan sanoa
johdannon väitteen totuudesta tai tutkimuskysymyksen vastauksesta.
Yhteenvedossa tuodaan myös esille tutkielman heikkoudet (erityisesti
tekijät, jotka heikentävät tutkielman tulosten luotettavuutta), ellei
niitä ole jo aiemmin tuotu esiin esimerkiksi Pohdinta-luvussa.  Tässä
luvussa voidaan myös tuoda esille, mitä tutkimusta olisi tämän
tutkielman tulosten valossa syytä tehdä seuraavaksi.

Jos Yhteenveto alkaa pitkittyä, se kannattaa jakaa kahtia niin, että
tulosten tulkinta otetaan omaksi Pohdinta-luvukseen, jolloin
Yhteenvedosta tulee varsin lyhyt ja lakoninen.

Yhteenvedon jälkeen tulee \string\printbibliography-komennolla
laadittu lähdeluettelo ja sen jälkeen mahdolliset liitteet.

\printbibliography

\appendix
\section{Siirtyminen gradu2:sta gradu3:een}

Keskeneräisen tutkielman siirtäminen gradu2:sta gradu3:een ei ole
kovin vaikeata.  Aluksi on totta kai vaihdettava
\string\documentclass-komennossa gradu2 gradu3:ksi.  Komennon
optioista suurin osa on poistettava, koska niitä ei enää tueta;
ainoastaan merkistön ilmoittava optio jää jäljelle.  Mahdollinen
kandi-optio vaihdetaan optioksi bachelor.

Taulukossa~\ref{tbl:cmdchange} on lueteltu tarvittavat
komentovaihdokset.  Viiva tarkoittaa, ettei vastaavaa komentoa ole
lainkaan.  Huomaa erityisesti uudet komennot.

\begin{table}[h]\centering
  \begin{tabular}{ll}
    \toprule
    gradu2                 & gradu3  \\
    \midrule
    ---                    & \string\maketitle \\
    ---                    & \string\supervisor \\
    \string\acmccs         & --- \\
    \string\aine           & \string\subject\\
    \string\copyrightowner & --- \\
    \string\fulltitle      & --- \\
    \string\laitos         & \string\department\\
    \string\license        & --- \\
    \string\linja          & \string\studyline\\
    \string\paikka         & --- \\
    \string\setauthor      & \string\author\\
    \string\termlist       & thetermlist-ympäristö\\
    \string\tyyppi         & \string\type\\
    \string\yhteystiedot   & \string\contactinformation\\
    \string\yliopisto      & \string\university\\
    \string\ysa            & --- \\
    \bottomrule
  \end{tabular}
  \caption{Komentomuutokset gradu2:sta gradu3:een}
  \label{tbl:cmdchange}
\end{table}

Isoin työ voi aiheutua lähdeluettelon laatimistekniikan muuttumiseen
sopeutumisesta.

\section{Harvemmin tarvittavat ominaisuudet}

Aiemmin esiteltyjen lisäksi gradu3 tarjoaa seuraavat lisäominaisuudet:
\begin{itemize}
\item \LaTeXe:n vakio-optiot draft ja final toimivat.
\item \LaTeXe:n vakio-optio twoside toimii myös.  Tätä voi käyttää
  esimerkiksi gradun kansitusversion laatimiseen, mutta virallisen
  arvostelu- ja arkistokappaleen laatimiseen sitä ei suositella.
\item Vaikka tutkielman suomenkielisyyttä ei tarvitse erikseen
  mainita, finnish-optio toimii.
\item \string\university-komennolla voit ilmoittaa tutkielman
  kotiyliopistoksi jonkin muun kuin Jyväskylän yliopiston.
\item  \string\department-komennolla voit ilmoittaa tutkielman
  kotilaitokseksi jonkin muun kuin Informaatioteknologian tiedekunnan.
\item \string\subject-komennolla voit ilmoittaa tutkielman
  oppiaineeksi jonkin muun kuin tietotekniikan.  Huomaa, että oppiaine
  tulisi suomenkielisissä tutkielmissa kirjoittaa genetiivimuodossa ja
  isolla alkukirjaimella (''Tietotekniikan''), englanninkielisissä
  tuktkielmissa in-preposition kanssa (''in Information Technology'').
\item \string\type-komennolla voit ilmoittaa tutkielman tyypin, jos se
  on jokin muu kuin pro gradu (oletus) tai kandidaatintutkielma
  (optiolla bachelor).
\item \string\setdate-komennolla voit asettaa päivämäärän
  haluamaksesi.  Anna komennolle kolme parametria -- päivä,
  kuukausi ja vuosi -- numeerisessa muodossa.
\item Ympäristöllä chapterquote voit laittaa luvun alkuun
  mietelauseen.  Sillä on yksi pakollinen parametri (lainauksen
  attribuutio).
\item Komento \string\graduclsdate\ sisältää käytössä olevan gradu3:n
  julkaisupäivämäärän ja \string\graduclsversion\ sen versionumeron.
\end{itemize}

\end{document}
