\begin{table}[htbp]
\centering
\caption{Comparison of granular damper configurations showing peak pressure reduction and energy dissipation effectiveness.}
\label{tab:damping_comparison}
\begin{tabular}{lrrrrr}
\hline
\textbf{Configuration} & \textbf{Mass} & \textbf{Diameter} & \textbf{Peak} & \textbf{Energy} & \textbf{Damping} \\
 & (kg) & (mm) & \textbf{Reduction (\%)} & \textbf{Dissip. (\%)} & \textbf{Ratio} \\
\hline
No damper (baseline) & 0.0 & 0.0 & 0.0 & 0.0 & 0.010 \\
Light damper (2\% mass) & 2.0 & 5.0 & 18.5 & 22.3 & 0.045 \\
Medium damper (5\% mass) & 5.0 & 5.0 & 34.2 & 41.8 & 0.089 \\
Heavy damper (10\% mass) & 10.0 & 5.0 & 52.1 & 63.4 & 0.142 \\
Fine particles (d=2mm) & 5.0 & 2.0 & 28.7 & 38.2 & 0.076 \\
Coarse particles (d=10mm) & 5.0 & 10.0 & 39.8 & 45.1 & 0.098 \\
\hline
\end{tabular}
\end{table}

\begin{table}[htbp]
\centering
\caption{Parametric study results showing optimal particle size and mass ratio for maximum damping.}
\label{tab:parametric_summary}
\begin{tabular}{lcc}
\hline
\textbf{Parameter} & \textbf{Optimal Value} & \textbf{Damping Ratio} \\
\hline
Particle diameter & 5 mm & 0.089 \\
Mass ratio & 10--15\% & 0.142--0.178 \\
Frequency tuning & 0.5 Hz & -- \\
\hline
\multicolumn{3}{l}{\textit{Note}: Optimal values based on 2m $\times$ 1m $\times$ 1m tank geometry.} \\
\end{tabular}
\end{table}
