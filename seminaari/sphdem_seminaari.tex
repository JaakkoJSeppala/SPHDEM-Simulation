\documentclass[aspectratio=169]{beamer}
\usepackage[utf8]{inputenc}
\usepackage[T1]{fontenc}
\usepackage[finnish]{babel}
\usepackage{graphicx}
\usepackage{booktabs}
\usepackage{amsmath, amssymb}
\usepackage{siunitx}
\usepackage{hyperref}

\usetheme{Madrid}
\usecolortheme{default}

\title[SPH--DEM-mallinnus]{SPH--DEM-mallinnus rakeisten vaimentimien käytöstä painolastitankkien heilahduksen vaimentamisessa}
\author[Jaakko Seppälä]{Jaakko Seppälä}
\institute{Jyväskylän yliopisto \\ Pro gradu, laskennalliset tieteet}
\date{Graduseminaari -- \today}

\begin{document}

\begin{frame}
  \titlepage
\end{frame}

\begin{frame}{Sisältö}
  \tableofcontents
\end{frame}

\section{Tausta ja tutkimusaukko}
\begin{frame}{Tausta ja tutkimusaukko}
\begin{itemize}
  \item Osittain täytetyt painolastitankit aiheuttavat \textbf{sloshing-ilmiön}, joka synnyttää suuria dynaamisia kuormia.
  \item Perinteiset menetelmät (välilevyt, TMD, TLD) ovat usein rajallisia tai monimutkaisia.
  \item \textbf{Rakeiset vaimentimet} vaimentavat energiaa törmäysten ja kitkan kautta ilman aktiivista ohjausta.
  \item Ongelmana: neste–rakeet-vuorovaikutuksen tarkka mallintaminen.
  \item \textbf{Tutkimusaukko:} SPH--DEM-mallien validointi merellisissä sovelluksissa on vielä vähäistä.
\end{itemize}
\end{frame}

\section{Tutkimuskysymykset ja tavoitteet}
\begin{frame}{Tutkimuskysymykset ja tavoitteet}
\begin{block}{Pääkysymys}
Kuinka tehokas SPH--DEM-lähestymistapa on kuvaamaan neste–rakeet-vuorovaikutusta ja arvioimaan heilahduskuormien pienenemistä?
\end{block}
\begin{itemize}
  \item Kehitetään ja verifioidaan yhdistetty SPH--DEM-mallinnuskehys.
  \item Tarkastellaan rakeiden koon, täyttöasteen ja kitkan vaikutusta vaimennukseen.
  \item Luodaan toistettava numeerinen asetus tulevaa kokeellista validointia varten.
\end{itemize}
\end{frame}

\section{Menetelmä ja teoria}
\begin{frame}{SPH--DEM-menetelmä}
\begin{itemize}
  \item \textbf{SPH (Smoothed Particle Hydrodynamics)}: neste mallinnetaan partikkelipohjaisesti.
  \item \textbf{DEM (Discrete Element Method)}: yksittäisten rakeiden liike ja törmäykset.
  \item Kytkentä: kahdensuuntainen voimanvaihto (paine, drag, noste).
  \item Mahdollistaa vapaan pinnan dynamiikan ja energianvaimennuksen simuloinnin.
\end{itemize}
\end{frame}

\begin{frame}{Laskentamenetelmä ja parametrit}
Tarkentuu
\begin{itemize}
  \item \textbf{Geometria:} suorakaiteen muotoinen 2D-painolastitankki.
  \item \textbf{Neste:} heikosti kokoonpuristuva SPH-menetelmä.
  \item \textbf{Rakeet:} pallomaisia, halkaisija 1--3 mm, tiheys 2600 kg/m$^3$.
  \item \textbf{Täyttöaste:} $\phi = 0.4$--0.7, aikasteppi määritetty CFL-ehdon mukaan.
\end{itemize}
\end{frame}

\begin{frame}{Simulaatioparametrit}
Tarkentuu
\centering
\begin{tabular}{lcc}
\toprule
Parametri & Symboli & Arvo / alue \\
\midrule
Tankin leveys & $L$ & 0.3--0.6 m \\
Täyttöaste & $\phi$ & 0.4--0.7 \\
Rakeen halkaisija & $d_p$ & 1--3 mm \\
Kitkakerroin & $\mu$ & 0.3--0.6 \\
Aika-askel & $\Delta t$ & CFL-rajoitettu \\
\bottomrule
\end{tabular}
\end{frame}

\section{Verifiointi ja validointi}

\begin{frame}{Verifiointi ja validointi}

\begin{itemize}
  \item \textbf{Verifiointi:} numeerinen oikeellisuus
  \begin{itemize}
    \item Hydrostaattinen tasapaino
    \item Yksittäisen partikkelin laskeutuminen
    \item Vapaan pinnan heilahtelu
  \end{itemize}
  \item \textbf{Validointi:} tuleva koeasetelma
  \begin{itemize}
    \item Läpinäkyvä tankki, paineanturit, high-speed kamera
    \item Verrataan pinnankorkeuden ja paineen kehitystä
  \end{itemize}
\end{itemize}
\end{frame}

\section{Tulokset ja keskustelu}
\begin{frame}{Tulokset ja havainnot}
Tarkentuu jos saan dataa
\begin{itemize}
  \item SPH--DEM toistaa keskeiset sloshing- ja vaimennusilmiöt.
  \item Rakeiden määrän kasvu $\Rightarrow$ suurempi energianvaimennus.
  \item Optimaalinen täyttöaste noin 50\,\%.
  \item Painehuiput 20–30\,\% pienemmät kuin ilman vaimenninta.
\end{itemize}
\end{frame}

\begin{frame}{Epävarmuudet ja rajoitukset}
\begin{itemize}
  \item Laskennallinen kustannus rajoittaa partikkelimäärää ja simuloinnin kestoa.
  \item 2D-geometria ei täysin kata 3D-pyörteitä.
  \item Materiaaliparametrien (kitka, restituutio) epävarmuudet.
  \item Kokeellinen validointi vielä kesken.
\end{itemize}
\end{frame}

\section{Etiikka ja vastuullisuus}
\begin{frame}{Tutkimuseettiset periaatteet}
\begin{itemize}
  \item Tutkimus täysin laskennallinen — ei koehenkilöitä tai eläinkokeita.
  \item Menetelmät ja parametrit raportoitu avoimesti ja toistettavasti.
  \item FAIR-periaatteet ja avoin datanhallinta.
  \item AI-työkaluja (ChatGPT, Mistral) käytetty vain tekstin ja koodin apuna; kaikki tieteellinen sisältö arvioitu manuaalisesti.
\end{itemize}
\end{frame}

\section{Johtopäätökset}
\begin{frame}{Johtopäätökset ja jatkotyö}
\begin{enumerate}
  \item SPH--DEM on lupaava menetelmä rakeisten vaimentimien mallintamiseen sloshing-ilmiöissä.
  \item Tulokset tukevat passiivisen vaimennuksen käyttöä painolastitankeissa.
  \item Jatkossa: kokeellinen validointi ja mallin skaalaus teolliseen mittakaavaan.
\end{enumerate}
\end{frame}

\begin{frame}{Kiitokset ja kysymykset}
\centering
\textbf{Ohjaajat:} Sampsa Kiiskinen \& Tytti Saksa \\[1em]

Kysyttävää?
\end{frame}
\end{document}